\documentclass[brudnopis]{xmgr}

%\defaultfontfeatures{Scale=MatchLowercase}
%\setmainfont[Numbers=OldStyle,Ligatures=TeX]{Minion Pro}
%\setsansfont[Numbers=OldStyle,Ligatures=TeX]{Myriad Pro}
% for fontspec version < 2.0
\setmainfont[Numbers=OldStyle,Mapping=tex-text]{Minion Pro}
\setsansfont[Numbers=OldStyle,Mapping=tex-text]{Myriad Pro}
%\setmonofont[Scale=0.75]{Monaco}
\usepackage{hyperref}
% Opcjonalnie identyfikator dokumentu 
% drukowany tylko z włączoną opcją 'brudnopis':
\wersja   {wersja wstępna [\ymdtoday]}

\author   {Oskar Plichta}
\nralbumu {195009}
\email    {oskar.plichta22@gmail.com}


\title    {Tworzenie przyjaznego interfejsu użytkownika w aplikacjach do udostępniania fotografii }
\date     {2014}
\miejsce  {Gdańsk}

\opiekun  {dr W.Bzyl}

% dodatkowe polecenia
%\renewcommand{\filename}[1]{\texttt{#1}}

\begin{document}

\begin{abstract}

W pracy zostanie przedstawiony przyjazny dla użytkownika interfejs aplikacji internetowej w web 3.0. Interfejs zostanie zaimplementowany w programie PicDrop,  który będzie spełniał założenia przyjaznego interfejsu. Celem aplikacji jest zbudowanie intuicyjnego interfejsu użytkownika do udostępniania fotografii wykonanego w AngularJS, Bootstrap oraz Ruby on Rails. Aplikacja ma na celu być prosta w obsłudze i ma za zadainie wyszukiwanie fotografi oraz ich łatwe udostępnianie dla wielu użytkowników jednocześnie. Ostatecznie aplikacja została wykonana zgodnie z założeniami i spełnia wyznaczone cele.

\end{abstract}
\keywords{User Interface, Ruby on Rails,Bootstrap, AngularJS, MongoDB, WebSockets, RSpec}

% tytuł i spis treści
\maketitle
%
% wstęp
\introduction
Interfejs użytkownika \footnote{ang. User Interface - UI}  jest podstawowym sposobem komunikacji pomiędzy człowiekiem a maszyną dlatego tak ważne jest, aby był on intuicyjny i przyjazny dla użytkownika. 
Postaram się pokazać na czym polega tworzenie przyjaznego UI i na czym ta przyjazność ma polegać.
Wskażę również z jakimi problemami musi się uporać developer aplikacji webowych, aby jego aplikacja była intuicyjna i funkcjonalna. Problem przyjaznego UI bardzo mnie zainteresował i dlatego postanowiłem zgłębić ten temat. Opierając się na  doświadczeniach innych badaczy  między innymi Roberta Hoekmana  \cite {magiaUI} oraz Jenifer Tidwell  \cite {projektowanieUI}, którzy opisali swoje spostrzeżenia w ich książkach, postaram się napisać aplikacje PicDrop, która  będzie miała przyjazne UI. Opiszę dlaczego wybrałem AngularJS, Bootstrap oraz Ruby on Rails do stworzenia tej aplikacji i dlaczego te a nie inne technologie uważam za najlepszy wybór.


\chapter{Kierunki rozwoju interfejsów użytkownika}

\section{Wprowadzenie do interfejsów użykownika}

\section{Nowe sposoby udostępniania treści}

\section{Przyjazne i intuicyjne interfejsy użytkownika w web 3.0}

\section{Przyjazne i intuicyjne interfejsy użytkownika w tabletach}

\section{Przyjazne i intuicyjne interfejsy użytkownika w smartphonach}

\chapter{Projekt UI dla aplikacji PicDrop}

\section{Przyjazność i intuicyjność  UI w aplikacji Picdrop}

\section{Udostępnianie treści w aplikacji Picdrop}

\chapter{Aplikacja do wyszukiwania i udostępniania zdjęć PicDrop}

Spośród mnóstwa technologii do tworzenia interfejsów użytkownika, najbardziej przodujące są oparte te na języku JavaScript takie jak Bootstrap, jQuery czy AngularJS. Opierając się na \cite{adamAnderson} oraz \cite{angularRailsBootstrap} postanowiłem wybrać AngularJS, Bootstrap oraz Ruby on Rails. UI zostanie wykonane w AngularJS, jest to stworzona i cały czas usprawniana przez firmę Google, biblioteka języka JavaScript. Posiada ona szereg mechanizmów ułatwiających developerom tworzenie UI na jej podstawie, ma także czytelną i przejrzystą dokumentację. UI aplikacji zostanie dodatkowo upiększone poprzez framework Bootstrap. Na serwer  wybrałem sprawdzonego Ruby on Rails, które pozwala na szybkie tworzenie aplikacji internetowych oraz dzięki narzędziu RSpec na łatwe testowanie kodu. 

\section{Cele aplikacji}

\section{Funkcjonowanie aplikacji}

\section{Opis tworzenia aplikacji PicDrop}

\section{Opis własnych rozwiązań}   

% zakończenie 
\summary

% załączniki (opcjonalnie):
\appendix
\chapter{Tytuł załącznika jeden}

Treść załącznika jeden.
\chapter{Tytuł załącznika dwa}

Treść załącznika dwa.

% literatura (obowiązkowo):
\bibliographystyle{unsrt}
\bibliography{mgr}

\nocite {magiaUI}
\nocite {design}
\nocite {projektowanieUI}
\nocite {StiveKrug}
\nocite {UIRails}
\nocite {rails4angular}
\nocite {modelingUI}
\nocite{adamAnderson}
\nocite{angularRailsBootstrap}
\nocite{angularRails}
\nocite {angularDoc}
\nocite{bootstrapDoc}
\nocite {mongoDoc}
\nocite {rspecDoc}
\nocite {railsDoc}

% spis tabel (jeżeli jest potrzebny):
\listoftables

% spis rysunków (jeżeli jest potrzebny):
\listoffigures

\oswiadczenie

\end{document}