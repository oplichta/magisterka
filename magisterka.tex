\documentclass[brudnopis]{xmgr}

%\defaultfontfeatures{Scale=MatchLowercase}
%\setmainfont[Numbers=OldStyle,Ligatures=TeX]{Minion Pro}
%\setsansfont[Numbers=OldStyle,Ligatures=TeX]{Myriad Pro}
% for fontspec version < 2.0
\setmainfont[Numbers=OldStyle,Mapping=tex-text]{Minion Pro}
\setsansfont[Numbers=OldStyle,Mapping=tex-text]{Myriad Pro}
%\setmonofont[Scale=0.75]{Monaco}
\usepackage{hyperref}
% Opcjonalnie identyfikator dokumentu 
% drukowany tylko z włączoną opcją 'brudnopis':
\wersja   {wersja wstępna [\ymdtoday]}

\author   {Oskar Plichta}
\nralbumu {195009}
\email    {oskar.plichta22@gmail.com}


\title    {Budowa aplikacji modularnej do udostępniania fotografii  w web 3.0}
\date     {2015}
\miejsce  {Gdańsk}

\opiekun  {dr W. Bzyl}

% dodatkowe polecenia
%\renewcommand{\filename}[1]{\texttt{#1}}

\begin{document}

\begin{abstract}

W pracy zostanie przedstawiony program do udostępniania fotografii na kilka serwisów społecznościowych jednocześnie z intuicyjnym interfejsem Material Design. Dzięki temu w prosty i szybki sposób można podzielić się swoimi zdjęciami z innymi użytkownikami kilku sieci społecznościowych. Aplikacja będzie wykonana z dwóch modułów - części serwerowej i wizualnej. Część serwerowa jest oparta o \textit{Ruby on Rails} a wizualna wykonana w \textit{EmberJS} oraz \textit{Materialize}. Aplikacje mobilna dla systemu iOS oraz Android wykonane zostały dzięki aplikacji Cordova, która pozwala przekształcić aplikację opartą o \textit{EmberJS} oraz \textit{Ember CLI} w natywną aplikacje dla danego systemu mobilnego. Ostatecznie aplikacja została wykonana zgodnie z założeniami i spełnia wyznaczone cele.

\end{abstract}
\keywords{\textit{User Interface, Material Design, Ruby on Rails, EmberJS, PostgreSQL, RSpec, Jasmine}}

% tytuł i spis treści
\maketitle
%
% wstęp
\introduction
Fotografie są jednym z najczęstszych typów danych przesyłanych w Web 3.0. Gwałtowny rozrost sieci społecznościowych spowodował, że prawie każdy udostępnia zdjęcia aby podzielić się nimi z rodziną i przyjaciółmi. Portale takie jak Facebook, Flickr czy Twitter prześcigają się w tym aby wysyłanie zdjęć na ich serwer było jak najprostsze. Większość z nich pozwala na tzw. drag and drop\footnote{ang. \textit{drag and drop} - przeciągnij i upuść} fotografii oraz na wysyłanie ich do innych serwisów. Zakładając, że mamy zdjęcia z wakacji i chcemy je udostępnić na Facebooku oraz umieścić na naszym koncie Flickr w celu archiwizacji musimy zalogować się na Facebooka, następnie wysłać zdjęcia na serwer Facebooka, ewentualnie dopisać opis i kliknąć w przycisk do udostępniania, po czym całość powtórzyć na serwisie Flickr. Temat wysyłania zdjęć do kilku serwisów jednocześnie, tak aby nie jeszcze raz nie powtarzać tej samej czynności lecz na innym serwisie społecznościowym, pozostaje otwarty i dlatego postanowiłem go zgłębić. 
Aplikacja, którą opisuje w tej pracy pozwala na jednoczesne wysyłanie zdjęć na kilka serwisów społecznościowych za pomocą kilku kliknięć. Wystarczy wybrać zdjęcia dzięki drag and drop, zalogować się na wybrane przez nas serwisy a następnie kliknąć wyślij. Wszystko przebiega szybko i sprawnie a my oszczędzamy nasz czas.  Aplikacja ta jest tzw. aplikacją modułową co znaczy, że składa się z  niezależnych od siebie części tj. serwerowej oraz wizualnej. Obie części komunikują się ze sobą za pomocą wiadomości JSON. Część serwerowa odpowiada za komunikację z bazą danych, komunikację z serwerami zewnętrznymi oraz autoryzację użytkowników i wysyłanie danych do front-endu czyli aplikacji wizualnej. Zostanie ona wykonana w języku Ruby i frameworku \textit{Ruby on Rails}.   Aplikacja wizualna, która zostanie wykonana w języku JavaScript i frameworku \textit{EmberJS}, ma za zadanie wyświetlanie danych w przystępnej formie dla użytkownika poprzez tzw. interfejs. Interfejs użytkownika \footnote{ang. \textit{User Interface} - UI}  jest podstawowym sposobem komunikacji pomiędzy człowiekiem a maszyną dlatego tak ważne jest, aby był on intuicyjny i przyjazny dla użytkownika. Postaram się pokazać dlaczego UI w mojej aplikacji jest przyjazny oraz intuicyjny dla użytkownika i pozwala mu na wydajną pracę.Większość smartphonów posiada dobrej jakości aparaty, zarówno z przodu jak i z tyłu urządzenia, dlatego w każdej chwili możemy wysłać zdjęcia z dowolnego miejsca do rodziny i przyjaciół poprzez sieci społecznościowe, za pomocą natywnej aplikacji na dany system mobilny. Dzięki temu, że moja aplikacja jest modułowa, jej część serwerowa czyli API pozwala na połączenie z nią różnych części wizualnych tzw. front-end z systemów mobilnych takich jak np. iOS lub Android. Jest to możliwe dzięki aplikacji Cordova, która zamienia aplikację \textit{EmberJS} w natywną aplikację na dany system mobilny przez co jest ona łatwiejsza w obsłudze i lepiej wykorzystuje mały wyświetlacz smartphona niż aplikacja przeglądarkowa. Wskażę również z jakimi problemami musi się uporać \textit{developer} aplikacji webowych, aby jego aplikacja była intuicyjna i funkcjonalna. Opierając się na  doświadczeniach innych badaczy  między innymi Roberta Hoekmana jr  \cite {magiaUI} oraz Jenifer Tidwell  \cite {projektowanieUI}, którzy opisali swoje spostrzeżenia w ich książkach, postaram się napisać aplikacje PicDrop, która  będzie miała przyjazne UI i pozwoli na łatwe udostępnianie treści. Opiszę dlaczego wybrałem \textit{EmberJS}, \textit{Bootstrap} oraz \textit{Ruby on Rails} do stworzenia tej aplikacji i dlaczego te technologie uważam za najlepszy wybór.


\chapter{Budowa aplikacji modularnej do udostępniania fotografii  w web 3.0}

\section{Porównanie dostępnych rozwiązań}
Każdy serwis społecznościowy posiada swoje możliwości udostępniania zdjęc, które ograniczają sie do jednej strony naraz. Tak więc jeśli chcemy wgrać swoje zdjęcia jednocześnie na Facebooka oraz Flickr to musimy je wgrać na jeden z tych dwóch serwisów a następnie na kolejny. Takie rozwiązanie zajmuje dużo czasu i jest niekorzystne dla użytkownika.
\section{Możliwości zastosowania praktycznego}
Głównym celem aplikacji PicDrop jest proste i intuicyjne udostępnianie fotografii. Użytkownik może wgrać własne fotografie poprzez przeciągniecie do okna przeglądarki. Następnie wystarczy, że użytkownik kliknie guzik z symbolem portalu gdzie chce udostępnić swoje fotografie i zaloguje sie na wybrany portal społecznościowy. To wszystko. Obsługa jest szybka i bezproblemowa. 


\chapter{Projekt i analiza}
\section{Aktorzy i przypadki użycia, wymagania funkcjonalne i niefunkcjonalne}
Aplikacja wczytuje zdjęcia z bazy danych PostgreSQL, wysyła je przez JSON do klienta i tam EmberJS odpowiednio obrabia dane pokazując je w formie przyjaznej użytkownikowi. Następnie, gdy chcemy udostępnić jakiś plik to jest to obsługiwane także przez EmberJS, tak aby komunikacja była szybka i niezawodna. Jednakże jeśli użytkownik wczytuje własne zdjęcia, które chce udostępnić to są one zamieniane na ciąg znaków w standardzie Base64Image i wysyłane wiadomością JSON do API. Tam back-end łączy się z wybranym portalem społecznościowym i przesyła do niego zdjęcia. 
\section{Diagram klas}
\section{Diagram modelu danych}
\section{Projekt interfejsu użytkownika w oparciu o framework Materialize}
\subsection{Wytyczne odnośnie UI w Material Design}

\chapter{Implementacja aplikacji PicDrop}
\section{Architektura aplikacji PicDrop}

Aplikacja składa się z dwóch części. Pierwszą z nich jest serwer API, które zostało stworzone w technologi Ruby on Rails w oparciu o gem rails-api. API jest pośrednikiem między bazą danych PostgreSQL a drugą częścią aplikacji czyli klientem stworzonym w EmberJS, który jest odpowiedzialny za warstwę wizualną aplikacji. Obie części komunikują się poprzez  JSON. Dzięki takiemu rozwiązaniu można stworzyć dodatkowych klientów, na przykład do aplikacji mobilnych.

\section{Użyte technologie}
Część serwerowa aplikacji może zostac wykonana w wielu językach programowania takich jak Ruby, Python, JavaScript czy PHP.
Wybrałem język Ruby, ponieważ jest on szybkim językiem skryptowym oraz posiada stabilny i sprawdzony framework \textit{Ruby on Rails}, który pozwala na szybkie tworzenie aplikacji internetowych oraz dzięki narzędziu \textit{RSpec} na łatwe testowanie kodu. 
Spośród mnóstwa technologii do tworzenia interfejsów użytkownika, najbardziej przodujące są oparte te na języku \textit{JavaScript} takie jak \textit{Bootstrap},\textit{ jQuery} czy\\ \textit{AngularJS}. Opierając się na artykułach \cite{} oraz \cite{} postanowiłem wybrać \textit{EmberJS}, \textit{Materialize} oraz \textit{Ruby on Rails}. UI zostanie wykonane w \textit{EmberJS}, jest to  biblioteka  \textit{open-source} języka \textit{JavaScript} stworzona  przez  Yehuda Katz'a  Tom Dale'a, która jest cały czas usprawniana przez społeczność. Posiada ona szereg mechanizmów ułatwiających developerom tworzenie UI na jej podstawie, ma także czytelną i przejrzystą dokumentację. UI aplikacji zostanie dodatkowo upiększone poprzez \textit{framework Materialize}, a \textit{Jasmine} pozwala na testowanie kodu w JavaScript.

\section{API aplikacji}
\subsection{Budowa API w oparciu o Ruby on Rails}
\subsection{Połączenie z bazą danych PostgreSQL}
\subsection{Dodawanie gemów w Ruby on Rails}
\subsection{Połączenie z API różnych sieci społecznościowych}
\subsection{Autoryzacja użytkowników}

\section{Front-end aplikacji}

\subsection{Budowa Front-endu w oparciu o framework EmberJS }
Warstwa wizualna aplikacji czyli tak zwany front-end został wykonany przy użyciu frameworka EmberJS. Dzięki temu łatwiej zarządzać tą cześcią aplikacji. EmberJS opiera się na wzorcu Model-View-Controler (pol. Model-Widok-Kontroler) co oznacza, że składa się z
trzech podstawowych części opowiedzialnych za różne akcje. Model  jest pewną reprezentacją problemu bądź logiki aplikacji. W naszym przypadku model jest szablonem z opisanymi typami danych danego zasobu np. zdjęcia. Model również komunikuje się z Ember Data czyli warstwą aplikacji bezpośrednio komunikującą się z serwerem.  Widok opisuje, jak wyświetlić pewną część modelu w ramach interfejsu użytkownika. W EmberJS widok składa się z tak zwanych templates czyli kodem HTML z aktywnie zmieniającymi się częściami. Kontroller przyjmuje dane wejściowe od użytkownika i reaguje na jego akcje, zarządzając aktualizacje modelu oraz odświeżenie widoków. Kontroler w EmberJS pobiera dane z modelu oraz zarządza akcjami w widoku.

\subsection{Połączenie z API}
Połączenie z serwerem aplikacji jest jedną z najważniejszych rzeczy w całej aplikacji. To dzięki niemu możemy się komunikować z serwerem oraz przetwarzac dane z bazy danych. EmberJS do pobierania i przetwarzania danych korzysta z biblioteki Ember Data. Ma ona kilka wbudowanych tzw. adapterów do połączenia z różnymi typami serwerów napisanych w różnych językach i technologiach. W mojej aplikacji używam ActiveModelAdapter, który działa bezproblemowo z ActiveModel w Ruby on Rails przeprowadzając serializacje i deserializację danych z EmberJS do wiadomości JSON.

\subsection{Opis narzędzia Ember-CLI}
Ember-CLI (Command Line Interface) to program narzędziowy, który zarządza aplikacją napisaną we frameworku EmberJS. Głównymi zaletami korzystania z tego narzędzia jest zarządzanie plikami, zależnościami, proste dodawanie wtyczek, uruchamianie serwera, generowanie konkretnych części programu jak np. model lub kontroler.

\subsection{Dodawanie wtyczek do aplikacji w EmberJS}
Wtyczki są to programy dodające jakąś funkcjonalność do naszego programu, dzięki czemu nie musimy wszystkich elementów pisać od podstaw. Wystarczy jak poszukamy wtyczki zapewniającej nam funkcjonalność, której szukamy np. t17-ember-upload pozwala na wgrywanie zdjęć do naszej aplikacji porzez proste Drag \& Drop (pol. Przeciągnij i upuść).  Aby dodać wtyczkę wystarczy zainstalować ją przy użyciu NPM lub bowera  po czym dodać app.import w pliku Brocfile.js. Następnie należy zaimportować konkretny moduł z wtyczki do naszej części aplikacji i już można używac widoku oraz akcji z wtyczki.

\section{Aplikacja mobilna  na system Android}
\subsection{Zalety aplikacji mobilnej względem strony w mobilnej przeglądarce}
\subsection{Tworzenie aplikacji przy użyciu narzedzia Cordova}

\chapter{Testy}
\section{Testowanie API przy użyciu RSpec}
\subsection{Scenariusz testowania}
\subsection{Raport z testów}
\section{Testowanie Front-end przy użyciu Jasmine}
\subsection{Scenariusz testowania}
\subsection{Raport z testów}
\chapter{Wkład własny}
% zakończenie 
\summary

% załączniki (opcjonalnie):
\appendix
\chapter{Tytuł załącznika jeden}

Treść załącznika jeden.
\chapter{Tytuł załącznika dwa}

Treść załącznika dwa.

% literatura (obowiązkowo):
\bibliographystyle{unsrt}
\bibliography{mgr}

\nocite {magiaUI}
\nocite {design}
\nocite {projektowanieUI}
\nocite {StiveKrug}
\nocite{DouglasCrockford}
\nocite {UIRails}
\nocite {modelingUI}
\nocite {rspecDoc}
\nocite {railsDoc}
\nocite {emberDoc}
\nocite{emberCLIDoc}
\nocite{emberRails}
\nocite{emberTutorial}
\nocite{emberIntro}

% spis tabel (jeżeli jest potrzebny):
\listoftables

% spis rysunków (jeżeli jest potrzebny):
\listoffigures

\oswiadczenie

\end{document}