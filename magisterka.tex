\documentclass[brudnopis]{xmgr}

%\defaultfontfeatures{Scale=MatchLowercase}
%\setmainfont[Numbers=OldStyle,Ligatures=TeX]{Minion Pro}
%\setsansfont[Numbers=OldStyle,Ligatures=TeX]{Myriad Pro}
% for fontspec version < 2.0
\setmainfont[Numbers=OldStyle,Mapping=tex-text]{Minion Pro}
\setsansfont[Numbers=OldStyle,Mapping=tex-text]{Myriad Pro}
%\setmonofont[Scale=0.75]{Monaco}

% Opcjonalnie identyfikator dokumentu 
% drukowany tylko z włączoną opcją 'brudnopis':
\wersja   {wersja wstępna [\ymdtoday]}

\author   {Oskar Plichta}
\nralbumu {195009}
\email    {oskar.plichta22@gmail.com}


\title    {Tworzenie przyjaznego interfejsu użytkownika i udostępnianie treści w aplikacjach internetowych}
\date     {2014}
\miejsce  {Gdańsk}

\opiekun  {dr W.Bzyl}

% dodatkowe polecenia
%\renewcommand{\filename}[1]{\texttt{#1}}

\begin{document}

\begin{abstract}
 Obecne na rynku serwisy do wyszukiwania takie jak Google Grafika, Flickr, Tumblr oraz Pinterest nie pozwalają na łatwe wyszukiwanie i udostępnianie zdjęć wybranych przez użytkownika a ich interfejsy użytkownika nie są mu przyjazne. Spośród bogatej oferty tworzenia UI zostaną wybrane i porównane interfejsy najbardziej intuicyjne i ułatwiające pracę użytkownikowi. Następnie zostanie stworzony nowy program, który będzie spełniał założenia przyjaznego dla użytkownika interfejsu w aplikacji internetowej.
\end{abstract}
\keywords{User Interface, Ruby on Rails,Bootstrap, AngularJS, MongoDB, WebSockets, RSpec}

% tytuł i spis treści
\maketitle
%
% wstęp
\introduction

\chapter{Wprowadzenie do Interfejsów Użytkownika}

\section{Rozwój interfejsów użytkownika w ostatnich latach – tendencje, kierunki rozwoju}

\section{Sposoby udostępniania treści\label{s:dtd}}

\section{Zdefinowanie przyjaznego i intuicyjnego interfejsu użytkownika}

\chapter{Przegląd dostępnych narzędzi do tworzenia UI }

\section{ Porównanie tworzenia aplikacji za pomocą dostępnych narzędzi\label{s:dsssl}}

\chapter{Opis wybranych narzędzi\label{s:xsl}}

\section{Ruby on Rails}

\section{AngularJS}

\section{Bootstrap}

\chapter{Budowa aplikacji do wyszukiwania i udostępniania zdjęć }

\section{Cele aplikacji}

\section{Funkcjonowanie aplikacji}

\chapter{Opis tworzenia aplikacji przy pomocy wybranych narzędzi\label{PRZEGLAD.NARZEDZI}}

\section{Opis tworzenia aplikacji PicDrop}

\section{Opis własnych rozwiązań}   

\chapter{Wnioski}

% zakończenie 
\summary

% załączniki (opcjonalnie):
\appendix
\chapter{Tytuł załącznika jeden}

Treść załącznika jeden.

\chapter{Tytuł załącznika dwa}

Treść załącznika dwa.

% literatura (obowiązkowo):
\bibliographystyle{unsrt}
\bibliography{xml}

\chapter{Literatura:}
Modeling the User Interface of Web Applications -http://www.pst.informatik.uni-muenchen.de/~kochn/pUML2001-Hen-Koch.pdf\\
Steve Krug - ”Don't Make Me Think ”\\
Don Norman- ”The Design of Everyday Things”\\
Jenifer Tidwell- „Projektowanie interfejsów. Sprawdzone wzorce projektowe”\\
Robert Hoekman jr -”Magia interfejsu. Praktyczne metody projektowania aplikacji internetowych”\\
How to integrate angularjs with rails -https://shellycloud.com/blog/2013/10/how-to-integrate-angularjs-with-rails-4\\
User Interface Thinking in Rails: An Example https://coderwall.com/p/zg4zfq\\
http://rubyonrails.org/documentation/\\
http://rspec.info/\\
https://docs.angularjs.org/api\\
http://docs.mongodb.org/manual/\\


% spis tabel (jeżeli jest potrzebny):
\listoftables

% spis rysunków (jeżeli jest potrzebny):
\listoffigures

\oswiadczenie

\end{document}