\documentclass[brudnopis]{xmgr}

%\defaultfontfeatures{Scale=MatchLowercase}
%\setmainfont[Numbers=OldStyle,Ligatures=TeX]{Minion Pro}
%\setsansfont[Numbers=OldStyle,Ligatures=TeX]{Myriad Pro}
% for fontspec version < 2.0
\setmainfont[Numbers=OldStyle,Mapping=tex-text]{Minion Pro}
\setsansfont[Numbers=OldStyle,Mapping=tex-text]{Myriad Pro}
%\setmonofont[Scale=0.75]{Monaco}
\usepackage{hyperref}
% Opcjonalnie identyfikator dokumentu 
% drukowany tylko z włączoną opcją 'brudnopis':
\wersja   {wersja wstępna [\ymdtoday]}

\author   {Oskar Plichta}
\nralbumu {195009}
\email    {oskar.plichta22@gmail.com}


\title    {Jak stworzyć przyjazny interfejs użytkownika do udostępniania fotografii w AngularJS? }
\date     {2014}
\miejsce  {Gdańsk}

\opiekun  {dr W.Bzyl}

% dodatkowe polecenia
%\renewcommand{\filename}[1]{\texttt{#1}}

\begin{document}

\begin{abstract}
 Obecne na rynku serwisy do wyszukiwania takie jak Google Grafika, Flickr, Tumblr oraz Pinterest nie pozwalają na łatwe wyszukiwanie i udostępnianie zdjęć wybranych przez użytkownika a ich interfejsy użytkownika nie są mu przyjazne. Spośród bogatej oferty tworzenia UI zostaną wybrane i porównane interfejsy najbardziej intuicyjne i ułatwiające pracę użytkownikowi. Następnie zostanie stworzony nowy program - PicDrop , który będzie spełniał założenia przyjaznego dla użytkownika interfejsu w aplikacji internetowej.
\end{abstract}
\keywords{User Interface, Ruby on Rails,Bootstrap, AngularJS, MongoDB, WebSockets, RSpec}

% tytuł i spis treści
\maketitle
%
% wstęp
\introduction

Interfejs użytkownika (ang. User Interface - UI ) jest podstawowym sposobem komunikacji pomiędzy człowiekiem a maszyną dlatego tak ważne jest, aby był on intuicyjny i przyjazny dla użytkownika. Praca w takim UI jest przyjemniejsza, szybsza i bardziej wydajna, ponieważ nie musimy się zastanawiać gdzie jest dana opcja programu czy strony lub jak uzyskać dany efekt. 

W aplikacjach internetowych interfejs użytkownika jest bardzo ważny, gdyż dzieki niemu użytkownik może w krókim czasie ocenić czy dana strona internetowa mu odpowiada i ewentualnie poszukać innej z lepszym UI. 
W tej pracy postaram się pokazać na czym polega tworzenie przyjaznego UI i na czym ta przyjazność ma polegać.
Wskażę również z jakimi problemami musi się uporać developer aplikacji webowych, aby jego aplikacja była intuicyjna i funkcjonalna. Opiszę również jakie narzędzia wybrałem i dlaczego te a nie inne uważam za najlepszy wybór.


\chapter{Wprowadzenie do Interfejsów Użytkownika}

\section{Rozwój interfejsów użytkownika w ostatnich latach – tendencje, kierunki rozwoju}

\section{Sposoby udostępniania treści\label{s:dtd}}

\section{Zdefinowanie przyjaznego i intuicyjnego interfejsu użytkownika}

\chapter{Budowa aplikacji do wyszukiwania i udostępniania zdjęć -PicDrop }

\section{Cele aplikacji}

\section{Funkcjonowanie aplikacji}

\section{Opis tworzenia aplikacji PicDrop}

\section{Opis własnych rozwiązań}   

\chapter{Przegląd dostępnych narzędzi do tworzenia UI }

\section{ Porównanie tworzenia aplikacji za pomocą dostępnych narzędzi\label{s:dsssl}}

\chapter{Opis wybranych narzędzi\label{s:xsl}}

\section{Ruby on Rails}

\section{AngularJS}

\section{Bootstrap}

\chapter{Wnioski}

% zakończenie 
\summary

% załączniki (opcjonalnie):
\appendix
\chapter{Tytuł załącznika jeden}

Treść załącznika jeden.
\chapter{Tytuł załącznika dwa}

Treść załącznika dwa.

% literatura (obowiązkowo):
\bibliographystyle{unsrt}
\bibliography{xml}

\nocite {StiveKurg}
\nocite {design}
\nocite {projektowanieUI}
\nocite {magiaUI}
\nocite {UIRails}
\nocite {rails4angular}
\nocite {modelingUI}
\nocite {angularDoc}
\nocite {mongoDoc}
\nocite {rspecDoc}
\nocite {railsDoc}

% spis tabel (jeżeli jest potrzebny):
\listoftables

% spis rysunków (jeżeli jest potrzebny):
\listoffigures

\oswiadczenie

\end{document}