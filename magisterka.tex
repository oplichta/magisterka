\documentclass[brudnopis]{xmgr}

%\defaultfontfeatures{Scale=MatchLowercase}
%\setmainfont[Numbers=OldStyle,Ligatures=TeX]{Minion Pro}
%\setsansfont[Numbers=OldStyle,Ligatures=TeX]{Myriad Pro}
% for fontspec version < 2.0
\setmainfont[Numbers=OldStyle,Mapping=tex-text]{Minion Pro}
\setsansfont[Numbers=OldStyle,Mapping=tex-text]{Myriad Pro}
%\setmonofont[Scale=0.75]{Monaco}
\usepackage{hyperref}
% Opcjonalnie identyfikator dokumentu 
% drukowany tylko z włączoną opcją 'brudnopis':
\wersja   {wersja wstępna [\ymdtoday]}

\author   {Oskar Plichta}
\nralbumu {195009}
\email    {oskar.plichta22@gmail.com}


\title    {Tworzenie przyjaznego interfejsu użytkownika w aplikacjach do udostępniania fotografii }
\date     {2014}
\miejsce  {Gdańsk}

\opiekun  {dr W.Bzyl}

% dodatkowe polecenia
%\renewcommand{\filename}[1]{\texttt{#1}}

\begin{document}

\begin{abstract}

W pracy zostanie przedstawiony przyjazny dla użytkownika interfejs aplikacji internetowej w web 3.0. Interfejs zostanie zaimplementowany w programie PicDrop,  który będzie spełniał założenia przyjaznego interfejsu. Celem aplikacji jest zbudowanie intuicyjnego interfejsu użytkownika do udostępniania fotografii wykonanego w \textit{EmberJS}, \textit{Bootstrap} oraz \textit{Ruby on Rails}. Aplikacja ma na celu być prosta \\ w obsłudze i ma za zadainie wyszukiwanie fotografi oraz ich łatwe udostępnianie dla wielu użytkowników jednocześnie. Ostatecznie aplikacja została wykonana zgodnie z założeniami i spełnia wyznaczone cele.

\end{abstract}
\keywords{\textit{User Interface, Ruby on Rails,Bootstrap, EmberJS, MongoDB, WebSockets, RSpec}}

% tytuł i spis treści
\maketitle
%
% wstęp
\introduction
Interfejs użytkownika \footnote{ang. \textit{User Interface} - UI}  jest podstawowym sposobem komunikacji pomiędzy człowiekiem a maszyną dlatego tak ważne jest, aby był on intuicyjny i przyjazny dla użytkownika. 
Postaram się pokazać na czym polega tworzenie przyjaznego UI, aby było intuicyjne dla użytkownika i pozwalało mu na wydajną pracę. Wskażę również z jakimi problemami musi się uporać \textit{developer} aplikacji webowych, aby jego aplikacja była intuicyjna i funkcjonalna. Serwisy do wyszukiwania zdjęć, nie mają dobrego UI i dlatego postanowiłem zgłębić ten temat. Opierając się na  doświadczeniach innych badaczy  między innymi Roberta Hoekmana jr  \cite {magiaUI} oraz Jenifer Tidwell  \cite {projektowanieUI}, którzy opisali swoje spostrzeżenia w ich książkach, postaram się napisać aplikacje PicDrop, która  będzie miała przyjazne UI i pozwoli na łatwe udostępnianie treści. Opiszę dlaczego wybrałem \textit{EmberJS}, \textit{Bootstrap} oraz \textit{Ruby on Rails} do stworzenia tej aplikacji i dlaczego te technologie uważam za najlepszy wybór.


\chapter{Kierunki rozwoju interfejsów użytkownika}

\section{Wprowadzenie do interfejsów użykownika}

\section{Nowe sposoby udostępniania treści}

\section{Przyjazne i intuicyjne interfejsy użytkownika\\ w web 3.0}

\section{Przyjazne i intuicyjne interfejsy użytkownika\\ w tabletach}

\section{Przyjazne i intuicyjne interfejsy użytkownika\\ w smartphonach}

\chapter{Projekt UI dla aplikacji PicDrop}

\section{Przyjazność i intuicyjność  UI w aplikacji PicDrop}

\section{Udostępnianie treści w aplikacji PicDrop}

\chapter{Aplikacja do wyszukiwania i udostępniania zdjęć PicDrop}

Spośród mnóstwa technologii do tworzenia interfejsów użytkownika, najbardziej przodujące są oparte te na języku \textit{JavaScript} takie jak \textit{Bootstrap},\textit{ jQuery} czy\\ \textit{AngularJS}. Opierając się na artykułach \cite{adamAnderson} oraz \cite{angularRailsBootstrap} postanowiłem wybrać \textit{EmberJS}, \textit{Bootstrap} oraz \textit{Ruby on Rails}. UI zostanie wykonane w \textit{EmberJS}, jest to  biblioteka  \textit{open-source} języka \textit{JavaScript} stworzona  przez  Yehuda Katz'a  Tom Dale'a, która jest cały czas usprawniana przez społeczność. Posiada ona szereg mechanizmów ułatwiających developerom tworzenie UI na jej podstawie, ma także czytelną i przejrzystą dokumentację. UI aplikacji zostanie dodatkowo upiększone poprzez \textit{framework Bootstrap}. Na serwer  wybrałem sprawdzonego \textit{Ruby on Rails}, które pozwala na szybkie tworzenie aplikacji internetowych oraz dzięki narzędziu \textit{RSpec} na łatwe testowanie kodu. 

\section{Cele aplikacji}

Głównym celem aplikacji PicDrop jest proste i intuicyjne udostępnianie fotografii. Użytkownik może wgrać własne fotografie lub wyszukac poprzez wbudowaną wyszukiwarkę. Następnie wystarczy, że użytkownik przeciągnie wybrane zdjęcia do bocznego paska i kliknie guzik z symbolem portalu gdzie chce udostępnić swoje fotografie. To wszystko. Obsługa jest szybka i bezproblemowa. 

\section{Funkcjonowanie aplikacji}

Aplikacja wczytuje zdjęcia z bazy danych MongoDB, wysyła je przez JSON do klienta i tam EmberJS odpowiednio obrabia dane pokazując je w formie przyjaznej użytkownikowi. Następnie, gdy chcemy udostępnić jakiś plik to jest to obsługiwane także przez EmberJS, tak aby komunikacja była szybka i niezawodna. 

\section{Opis tworzenia aplikacji PicDrop}

Aplikacja składa się z dwóch części. Pierwszą z nich jest serwer API, które zostało stworzone w technologi Ruby on Rails w oparciu o gem Grape. API jest pośrednikiem między bazą danych MongoDB a drugą częścią aplikacji czyli klientem stworzonym w EmberJS, który jest odpowiedzialny za warstwę wizualną aplikacji. Obie części komunikują się poprzez  JSON. Dzięki takiemu rozwiązaniu można stworzyć dodatkowych klientów, na przykład do aplikacji mobilnych.

\section{Opis własnych rozwiązań}   

% zakończenie 
\summary

% załączniki (opcjonalnie):
\appendix
\chapter{Tytuł załącznika jeden}

Treść załącznika jeden.
\chapter{Tytuł załącznika dwa}

Treść załącznika dwa.

% literatura (obowiązkowo):
\bibliographystyle{unsrt}
\bibliography{mgr}

\nocite {magiaUI}
\nocite {design}
\nocite {projektowanieUI}
\nocite {StiveKrug}
\nocite{DouglasCrockford}
\nocite {UIRails}
\nocite {modelingUI}
\nocite{bootstrapDoc}
\nocite {mongoDoc}
\nocite {rspecDoc}
\nocite {railsDoc}
\nocite {emberDoc}
\nocite{emberCLIDoc}
\nocite{emberRails}
\nocite{emberTutorial}
\nocite{emberIntro}

% spis tabel (jeżeli jest potrzebny):
\listoftables

% spis rysunków (jeżeli jest potrzebny):
\listoffigures

\oswiadczenie

\end{document}